\documentclass{article}
\usepackage[utf8]{inputenc}
\usepackage{graphicx}

\title{Rapport TP1 - Langage SLIP}
\author{Mathias La Rochelle \& Michel Lamothe}
\date{Le samedi 29 septembre 2024}

\begin{document}

\maketitle

\section{Introduction}
\subsection{Objectifs du TP}
    Afin d'avoir une session interactive sur GHCi avec aucuns avertissements et aucunes exceptions, nous allons devoir compléter
    les fonctions \textit{s2l} et \textit{eval}. Nous devons également écrire le code d'au moins 5 tests dans un fichier \texttt{tests.slip}. Tout doit bien s'exécuter
    évidemment. \\

    \noindent À noter que le tout doit bien s'exécuter sur les machines du laboratoire du DIRO à travers une connexion SSH avec \texttt{ens.iro}.
    \subsection{Présentation du langage SLIP}
    Le langage SLIP est un langage qui se rapproche fortement de la famille de langages Lisp. Il s'agit d'un langage fonctionnel
    conçu pour être simple et expressif, tout en conservant certaines caractéristiques clés de Lisp. SLIP utilise une syntaxe
    basée sur les expressions S. Nous allons explorer en profondeur le fonctionnement interne de SLIP en implémentant ses composants essentiels.

\subsection{Structure du rapport}
    Ce rapport est organisé de la manière suivante :
    \begin{itemize}
        \item La section 2 présente l'implémentation des fonctions principales, notamment \textit{s2l} et \textit{eval}.
        \item La section 3 détaille les exemples et les tests réalisés pour valider notre implémentation.
        \item La section 4 discute des difficultés rencontrées et des solutions apportées.
        \item La section 5 conclut le rapport en résumant nos réalisations et en proposant des pistes d'amélioration.
    \end{itemize}

    Tout au long de ce rapport, nous mettrons en évidence les concepts clés du langage SLIP et les choix d'implémentation
    effectués pour répondre aux exigences du TP.

\section{Implémentation}
\subsection{Structures de données utilisées}
\subsection{Fonctions principales}
\subsection{Difficultés rencontrées et solutions}

\section{Exemples et tests}
\subsection{Cas de test simples}
\subsection{Cas de test complexes}
\subsection{Analyse des résultats}

\section{Extension du langage}
\subsection{Nouvelles fonctionnalités implémentées}
\subsection{Justification des choix}

\section{Conclusion}
\subsection{Récapitulatif des réalisations}
\subsection{Perspectives d'amélioration}

\section{Annexes}
\subsection{Code source complet}
\subsection{Exemples supplémentaires}

\end{document}
