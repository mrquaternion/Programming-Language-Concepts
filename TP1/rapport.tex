\documentclass{article}
\usepackage[utf8]{inputenc}
\usepackage{graphicx}
\usepackage[a4paper, left=1in, right=1in, top=1in, bottom=1in]{geometry}
\usepackage{enumitem}
\usepackage{listings}

\lstset{ % Configuration des listings
  language=Haskell,               % Langage de programmation
  basicstyle=\ttfamily\footnotesize, % Style du texte
  numbers=left,                  % Numérotation des lignes à gauche
  stepnumber=1,                  % Numérotation des lignes toutes les 1 ligne
  showstringspaces=false,         % Ne montre pas les espaces dans les chaînes
  frame=single,                  % Cadre autour du code
}


\title{Rapport TP1}
\author{Mathias La Rochelle \& Michel Lamothe}
\date{Le samedi 29 septembre 2024}

\begin{document}

\maketitle
\setlength{\parindent}{0pt}


\section{Introduction}
\subsection{Objectifs du TP}
Afin d'avoir une session interactive sur GHCi avec aucuns avertissements et aucunes exceptions, 
nous allons devoir compléter les fonctions \textit{s2l} et \textit{eval}. Nous devons également 
écrire le code d'au moins 5 tests dans un fichier \texttt{tests.slip}. Tout doit bien s'exécuter 
évidemment. \\

Tout au long de ce rapport, nous mettrons en évidence les concepts clés du langage SLIP et les 
choix d'implémentation effectués pour répondre aux exigences du TP. 

\subsection{Difficultés rencontrées et solutions}
\begin{enumerate}
    \item \underline{Problème} : Extraction des arguments et liaisons récursives 
        \begin{itemize}
            \item \textit{Description :} Au début, il était difficile d’extraire correctement les variables et 
                                         leur corps pour \texttt{Lfix}.
            \item \textit{Solution :} L’utilisation de fonctions auxiliaires (bindExtraction et argsExtraction) 
                                      a permis de gérer efficacement les expressions imbriquées et de structurer 
                                      les arguments de manière cohérente.
        \end{itemize}
        
    \item \underline{Problème} : Gestion de l'environnement d’exécution
        \begin{itemize}
            \item \textit{Description :} Il était complexe de maintenir un suivi précis de l’environnement avec 
                                         des déclarations imbriquées. Par exemple, les variables définies dans un 
                                         ``let" ne doivent pas affecter l’environnement global.
            \item \textit{Solution :} Nous avons approché le problème en ajoutant temporairement des variables dans 
                                      l’environnement ce qui a permis de résoudre le problème sans modifier les 
                                      environnements globaux. L’utilisation de listes immuables pour représenter 
                                      l'environnement a également facilité la gestion.
        \end{itemize}

    \item \underline{Problème} : Gestion de l'environnement d’exécution
        \begin{itemize}
            \item \textit{Description :} La gestion des fonctions et objets fonctionnels définis avec \texttt{Lfob} posait 
                                         plusieurs défis. Contrairement à un simple appel de fonction,\texttt{Lfob} permet 
                                         de définir une fonction en capturant un environnement local. 
                                         Cela signifie que la fonction peut être appelée plus tard avec des arguments
                                         de l'environnement extérieur et en se basant sur les variables définies lors 
                                         de sa création. Cependant, plusieurs problèmes sont survenus :
                                        \begin{enumerate}
                                            \item \texttt{Correspondance des paramètres et arguments :} Il fallait 
                                                          s'assurer que le nombre de paramètres fournis lors de l’appel 
                                                          corresponde exactement au nombre d'arguments attendus par la fonction.
                                            \item \texttt{Gestion de l’environnement capturé :} L’environnement d’exécution 
                                                          lors de la définition de la fonction devait être stocké et réutilisé 
                                                          lorsque la fonction était appelée.
                                            \item \texttt{Appels imbriqués :} Dans certains cas, une fonction pouvait retourner 
                                                                              une autre fonction, ce qui nécessitait une gestion 
                                                                              des closures imbriquées.
                                            \item \texttt{Traitement des erreurs :} Une mauvaise correspondance entre les 
                                                          paramètres et les arguments devait être signalée immédiatement 
                                                          avec un message d'erreur explicite.
                                        \end{enumerate}
            \newpage
            \item \textit{Solution :} 
                \begin{enumerate}
                    \item \texttt{Validation du nombre d'arguments :} Lors de l’évaluation d’une fonction avec \texttt{Lsend}, une 
                                  vérification explicite a été ajoutée pour s'assurer que le nombre d'arguments correspond au nombre 
                                  de paramètres. En cas de déséquilibre, une erreur descriptive est levée:\newline
                    \begin{lstlisting}
if length params == length args
    then 
        ...
    else error ("Manque " ++ show (abs (length params - length args)) 
                ++ " arguments.")
                    \end{lstlisting}
                \end{enumerate}
        \end{itemize}
\end{enumerate}

\section{Exemples et tests} 
\subsection{Cas de test simples} 
\subsubsection*{Test 1 : Fonction sans paramètre} 
\hspace{.5cm}\textbf{Expression :} 
\begin{verbatim} 
    (fob () 42)) 
\end{verbatim} 
\hspace{.5cm}\textbf{Explications :} 
\begin{itemize}[left=1cm] 
    \item Teste la création et l'appel d'une fonction sans paramètre. 
    \item Vérifie que la fonction retourne bien la valeur 42. 
\end{itemize}
\subsubsection*{Test 2 : Fonction avec paramètre simple} 
\hspace{.5cm}\textbf{Expression :} 
\begin{verbatim} 
    ((fob (x) (if (= x 5) true false)) 5) 
\end{verbatim} 
\hspace{.5cm}\textbf{Explications :} 
\begin{itemize}[left=1cm] 
    \item Teste une fonction qui prend un paramètre. 
    \item Vérifie si le paramètre vaut 5 pour retourner \texttt{true}, sinon \texttt{false}. 
\end{itemize}
\subsubsection*{Test 3 : Fonction avec let et fermeture} 
\hspace{.5cm}\textbf{Expression :} 
\begin{verbatim} 
    (let x 5 ((fob (x) (+ x 3)) x)) 
\end{verbatim} 
\hspace{.5cm}\textbf{Explications :} 
\begin{itemize}[left=1cm] 
    \item Déclare une variable \texttt{x} locale avec la valeur 5. 
    \item Appelle une fonction qui ajoute 3 à \texttt{x} et retourne le résultat. 
\end{itemize}

\subsection{Cas de test complexes}
\subsubsection*{Test 4 : Fonction récursive (factorielle)} 
\hspace{.5cm}\textbf{Expression :} 
\begin{verbatim} 
    (fix ((fact (fob (n) 
                        (if (= n 0) 1 
                        (* n (fact (- n 1))))))) 
         (fact 5)) 
\end{verbatim} 
\hspace{.5cm}\textbf{Explications :} 
\begin{itemize}[left=1cm]
    \item Teste la récursion pour calculer la factorielle de 5. 
    \item Vérifie que la base de récursion (\texttt{n = 0}) retourne 1. 
\end{itemize}
\subsubsection*{Test 5 : Fonction imbriquée avec trois paramètres} 
\hspace{.5cm}\textbf{Expression :} 
\begin{verbatim} 
    (let volume ((((fob (x) (fob (y) (fob (z) (* x (* y z))))) 3) 2) 4) 
                (if (= volume 16) true false))
\end{verbatim} 
\hspace{.5cm}\textbf{Explications :} 
\begin{itemize}[left=1cm]
    \item Teste une fonction imbriquée prenant trois paramètres (\texttt{x}, \texttt{y}, \texttt{z}). 
    \item Calcule \texttt{volume} et vérifie si la valeur obtenue est 16. 
\end{itemize}

\section{Extension du langage}
\subsection{Nouvelles fonctionnalités implémentées}
\subsubsection*{Variable contenant des chaînes de caractères et opérations sur elles}
Pour intégrer la gestion des chaînes de caractères, il serait pertinent d'ajouter un nouveau type de données dans notre structure Lexp pour représenter les chaînes. Dans la fonction d'évaluation ``eval", il conviendrait d'ajouter des cas spécifiques pour traiter les opérations prédéfinies sur les chaînes comme la concaténation et la comparaison. Des fonctions de manipulation des chaînes pourraient aussi être incorporées dans ``env0", de manière similaire à l'intégration des autres opérateurs.

\subsubsection*{Les fonctions anonymes lambda}
Une autre extension intéressante serait l'ajout de la prise en charge des lambdas. En ajoutant une nouvelle construction dans ``Lexp" pour représenter des fonctions sans leur donner de nom, nous pourrions permettre à l'utilisateur de définir des fonctions directement dans le code. Dans la fonction d'évaluation ``eval", les lambdas seraient évalués en tant qu'objets de fonction 
avec des environnements de fermeture capturant les variables nécessaires. Utiliser un lambda évitera d'encombrer le code avec 
des définitions de fonctions supplémentaires et permettera de rendre ce langage plus flexible.

\section{Conclusion}
En conclusion, l'implantation de Lisp nous a permis de développer un analyseur lexical et syntaxique, ainsi 
qu'un évaluateur pour diverses expressions arithmétiques et logiques. À noter que ce fichier \LaTeX\ compile également sur \texttt{ens.iro} par SSH.

\end{document}

