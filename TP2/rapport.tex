\documentclass[a4paper,12pt]{article}

% Packages nécessaires
\usepackage[utf8]{inputenc}
\usepackage[T1]{fontenc}
\usepackage[french]{babel}
\usepackage{amsmath,amssymb}
\usepackage{graphicx}
\usepackage{geometry}
\usepackage{hyperref}
\usepackage{fancyhdr}
\usepackage{enumitem}
\usepackage{titlesec}

% Configuration des marges
\geometry{top=2.5cm, bottom=2.5cm, left=2.5cm, right=2.5cm}

% Titres des sections
\titleformat{\section}{\large\bfseries}{\thesection}{1em}{}
\titleformat{\subsection}{\normalsize\bfseries}{\thesubsection}{1em}{}

% En-têtes et pieds de page
\pagestyle{fancy}
\fancyhf{}
\fancyhead[L]{IFT-2035 : TP2}
\fancyhead[R]{\thepage}
\fancyfoot[C]{\small Université de Montréal \textbullet{} \today}

% Définition du titre et des auteurs
\title{IFT-2035 : TP2}
\author{
    Mathias La Rochelle (20269985) \\ 
    Michel Lamothe (XXXXXXXX)
}
\date{\today}

\begin{document}

% Page de titre
\maketitle

\vspace{2em}
% Table des matières
\tableofcontents
\newpage

% Contenu principal
\section{Problèmes rencontrés}
    \subsection{\texttt{s2l}}
        La fonction \texttt{s2l} n'a pas été un très gros problème. La grande majorité du code
        provient du solutionnaire donné sur StudiUM. Les cas qui ont dû être adaptés afin de 
        prendre en compte l'ajout des types sont les expressions de déclarations de variables
        avec `fix' et les déclarations de fonctions avec `fob'. Pour `fix', ce qui a été le 
        plus énervant était la gestion des deux types d'écriture : celle avec la précision
        des types des arguments et celle qui ajoutait la précision du type de la valeur de 
        retour. Cependant, la fonction auxiliaire que nous avons écris, i.e. \texttt{argsToTuple}, 
        a pu être utilisée non seulement pour ces deux cas de `fix', mais également le 
        pattern-match avec `fob'. Sa réutilisation a permis d'alléger l'ensemble du code qui
        était, au début, était lourd et peu lisible.
    \subsection{\texttt{check}}
    \subsection{\texttt{l2d}}
        La fonction \texttt{l2d} a été compliqué à implémenter. La suppression des annotations
        de types n'était pas facile car il a fallu plusieurs fois créer des nouveaux environnements
        pour garder en compte les variables/arguments ainsi que leur type. Particulièrement, lors
        de l'écriture des cas pour `Lsend' et `Lfix', nous nous sommes retrouvés régulièrement à
        écrire sur une feuille de papier ce qui se passait réellement dans de telles expressions
        lors du renommage. Il fallait anticiper ce qui allait dérouler lors de l'évaluation future
        des \texttt{Dexp}.
    \subsection{\texttt{eval}}

\newpage
\section{Structures des tests}
    \subsection{Typés correctement}
    \subsection{Refusés par le vérificateur}
        \subsubsection{Échoue sur \texttt{eval}}
        \subsubsection{N'échoue pas}

\newpage
\section{Extensions possibles}
    \subsection{Types supplémentaires}
        Afin d'enrichir le language \texttt{sslip} et d'en augmenter ses capacités, l'ajout de 
        nouveaux constructeurs au type \texttt{Type} serait une extension qui pourrait être 
        considérée.
        \subsubsection{\texttt{Tstring}}
            Les chaînes de caractères ne sont pas souvent priorisées dans la création de nouveaux
            languages de programmation mais cela est dû au fait qu'elles sont complexes à gérer.
            Cependant, leur implémentation pourrait être fait de manière plus simple si un type
            pour la gestion de listes serait implémenté. Alors, on aurait uniquement besoin 
            d'ajouter un type \texttt{Tchar} étant donné qu'une chaîne de caractères se retrouve
            à être simplement une liste de \texttt{Char}. De plus, cet ajencement serait plus
            flexible au niveau des opérations de chaînes telles que des concaténations ou des 
            comparaisons seraient plus facile à implémenter. Il y aurait quelques changements 
            à faire dans les "parser" de l'interpréteur, mais après le reste sera dirigé par 
            des règles dans les fonctions que nous avons définies.
        \subsubsection{\texttt{Tlist}}
            Un tel constructeur permettrait d'effectuer des manipulations sur des listes
            de différents types de valeurs. Les listes sont très importantes dans les 
            languages de programmation et ici, cela rajoutterai un niveau de complexité 
            sans trop avoir à modifier le code. Comme précisé plus tôt, celles-ci permettraient
            également d'implanter les chaînes de caractères. On pourrait aussi laisser
            place à notre imagination et construire "from scratch" des structures de données
            telles que des piles ou des files d'attentes.
    \subsection{Boucles}
        Les boucles telles que \texttt{while} serait utile pour itérer à travers notre toute 
        nouvelle implémentation du type \texttt{Tlist}. Son ajout au language se ferait comme 
        toute autre expression : gestion du cas de rencontre `while' dans \texttt{s2l}, nouvelle
        règle dans le "parser", un nouveau constructure dans le type \texttt{Lexp} et \texttt{Dexp}
        et j'en passe.


\newpage
\section{Conclusion}

\end{document}